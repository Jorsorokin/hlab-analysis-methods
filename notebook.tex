
% Default to the notebook output style

    


% Inherit from the specified cell style.




    
\documentclass[11pt]{article}

    
    
    \usepackage[T1]{fontenc}
    % Nicer default font (+ math font) than Computer Modern for most use cases
    \usepackage{mathpazo}

    % Basic figure setup, for now with no caption control since it's done
    % automatically by Pandoc (which extracts ![](path) syntax from Markdown).
    \usepackage{graphicx}
    % We will generate all images so they have a width \maxwidth. This means
    % that they will get their normal width if they fit onto the page, but
    % are scaled down if they would overflow the margins.
    \makeatletter
    \def\maxwidth{\ifdim\Gin@nat@width>\linewidth\linewidth
    \else\Gin@nat@width\fi}
    \makeatother
    \let\Oldincludegraphics\includegraphics
    % Set max figure width to be 80% of text width, for now hardcoded.
    \renewcommand{\includegraphics}[1]{\Oldincludegraphics[width=.8\maxwidth]{#1}}
    % Ensure that by default, figures have no caption (until we provide a
    % proper Figure object with a Caption API and a way to capture that
    % in the conversion process - todo).
    \usepackage{caption}
    \DeclareCaptionLabelFormat{nolabel}{}
    \captionsetup{labelformat=nolabel}

    \usepackage{adjustbox} % Used to constrain images to a maximum size 
    \usepackage{xcolor} % Allow colors to be defined
    \usepackage{enumerate} % Needed for markdown enumerations to work
    \usepackage{geometry} % Used to adjust the document margins
    \usepackage{amsmath} % Equations
    \usepackage{amssymb} % Equations
    \usepackage{textcomp} % defines textquotesingle
    % Hack from http://tex.stackexchange.com/a/47451/13684:
    \AtBeginDocument{%
        \def\PYZsq{\textquotesingle}% Upright quotes in Pygmentized code
    }
    \usepackage{upquote} % Upright quotes for verbatim code
    \usepackage{eurosym} % defines \euro
    \usepackage[mathletters]{ucs} % Extended unicode (utf-8) support
    \usepackage[utf8x]{inputenc} % Allow utf-8 characters in the tex document
    \usepackage{fancyvrb} % verbatim replacement that allows latex
    \usepackage{grffile} % extends the file name processing of package graphics 
                         % to support a larger range 
    % The hyperref package gives us a pdf with properly built
    % internal navigation ('pdf bookmarks' for the table of contents,
    % internal cross-reference links, web links for URLs, etc.)
    \usepackage{hyperref}
    \usepackage{longtable} % longtable support required by pandoc >1.10
    \usepackage{booktabs}  % table support for pandoc > 1.12.2
    \usepackage[inline]{enumitem} % IRkernel/repr support (it uses the enumerate* environment)
    \usepackage[normalem]{ulem} % ulem is needed to support strikethroughs (\sout)
                                % normalem makes italics be italics, not underlines
    

    
    
    % Colors for the hyperref package
    \definecolor{urlcolor}{rgb}{0,.145,.698}
    \definecolor{linkcolor}{rgb}{.71,0.21,0.01}
    \definecolor{citecolor}{rgb}{.12,.54,.11}

    % ANSI colors
    \definecolor{ansi-black}{HTML}{3E424D}
    \definecolor{ansi-black-intense}{HTML}{282C36}
    \definecolor{ansi-red}{HTML}{E75C58}
    \definecolor{ansi-red-intense}{HTML}{B22B31}
    \definecolor{ansi-green}{HTML}{00A250}
    \definecolor{ansi-green-intense}{HTML}{007427}
    \definecolor{ansi-yellow}{HTML}{DDB62B}
    \definecolor{ansi-yellow-intense}{HTML}{B27D12}
    \definecolor{ansi-blue}{HTML}{208FFB}
    \definecolor{ansi-blue-intense}{HTML}{0065CA}
    \definecolor{ansi-magenta}{HTML}{D160C4}
    \definecolor{ansi-magenta-intense}{HTML}{A03196}
    \definecolor{ansi-cyan}{HTML}{60C6C8}
    \definecolor{ansi-cyan-intense}{HTML}{258F8F}
    \definecolor{ansi-white}{HTML}{C5C1B4}
    \definecolor{ansi-white-intense}{HTML}{A1A6B2}

    % commands and environments needed by pandoc snippets
    % extracted from the output of `pandoc -s`
    \providecommand{\tightlist}{%
      \setlength{\itemsep}{0pt}\setlength{\parskip}{0pt}}
    \DefineVerbatimEnvironment{Highlighting}{Verbatim}{commandchars=\\\{\}}
    % Add ',fontsize=\small' for more characters per line
    \newenvironment{Shaded}{}{}
    \newcommand{\KeywordTok}[1]{\textcolor[rgb]{0.00,0.44,0.13}{\textbf{{#1}}}}
    \newcommand{\DataTypeTok}[1]{\textcolor[rgb]{0.56,0.13,0.00}{{#1}}}
    \newcommand{\DecValTok}[1]{\textcolor[rgb]{0.25,0.63,0.44}{{#1}}}
    \newcommand{\BaseNTok}[1]{\textcolor[rgb]{0.25,0.63,0.44}{{#1}}}
    \newcommand{\FloatTok}[1]{\textcolor[rgb]{0.25,0.63,0.44}{{#1}}}
    \newcommand{\CharTok}[1]{\textcolor[rgb]{0.25,0.44,0.63}{{#1}}}
    \newcommand{\StringTok}[1]{\textcolor[rgb]{0.25,0.44,0.63}{{#1}}}
    \newcommand{\CommentTok}[1]{\textcolor[rgb]{0.38,0.63,0.69}{\textit{{#1}}}}
    \newcommand{\OtherTok}[1]{\textcolor[rgb]{0.00,0.44,0.13}{{#1}}}
    \newcommand{\AlertTok}[1]{\textcolor[rgb]{1.00,0.00,0.00}{\textbf{{#1}}}}
    \newcommand{\FunctionTok}[1]{\textcolor[rgb]{0.02,0.16,0.49}{{#1}}}
    \newcommand{\RegionMarkerTok}[1]{{#1}}
    \newcommand{\ErrorTok}[1]{\textcolor[rgb]{1.00,0.00,0.00}{\textbf{{#1}}}}
    \newcommand{\NormalTok}[1]{{#1}}
    
    % Additional commands for more recent versions of Pandoc
    \newcommand{\ConstantTok}[1]{\textcolor[rgb]{0.53,0.00,0.00}{{#1}}}
    \newcommand{\SpecialCharTok}[1]{\textcolor[rgb]{0.25,0.44,0.63}{{#1}}}
    \newcommand{\VerbatimStringTok}[1]{\textcolor[rgb]{0.25,0.44,0.63}{{#1}}}
    \newcommand{\SpecialStringTok}[1]{\textcolor[rgb]{0.73,0.40,0.53}{{#1}}}
    \newcommand{\ImportTok}[1]{{#1}}
    \newcommand{\DocumentationTok}[1]{\textcolor[rgb]{0.73,0.13,0.13}{\textit{{#1}}}}
    \newcommand{\AnnotationTok}[1]{\textcolor[rgb]{0.38,0.63,0.69}{\textbf{\textit{{#1}}}}}
    \newcommand{\CommentVarTok}[1]{\textcolor[rgb]{0.38,0.63,0.69}{\textbf{\textit{{#1}}}}}
    \newcommand{\VariableTok}[1]{\textcolor[rgb]{0.10,0.09,0.49}{{#1}}}
    \newcommand{\ControlFlowTok}[1]{\textcolor[rgb]{0.00,0.44,0.13}{\textbf{{#1}}}}
    \newcommand{\OperatorTok}[1]{\textcolor[rgb]{0.40,0.40,0.40}{{#1}}}
    \newcommand{\BuiltInTok}[1]{{#1}}
    \newcommand{\ExtensionTok}[1]{{#1}}
    \newcommand{\PreprocessorTok}[1]{\textcolor[rgb]{0.74,0.48,0.00}{{#1}}}
    \newcommand{\AttributeTok}[1]{\textcolor[rgb]{0.49,0.56,0.16}{{#1}}}
    \newcommand{\InformationTok}[1]{\textcolor[rgb]{0.38,0.63,0.69}{\textbf{\textit{{#1}}}}}
    \newcommand{\WarningTok}[1]{\textcolor[rgb]{0.38,0.63,0.69}{\textbf{\textit{{#1}}}}}
    
    
    % Define a nice break command that doesn't care if a line doesn't already
    % exist.
    \def\br{\hspace*{\fill} \\* }
    % Math Jax compatability definitions
    \def\gt{>}
    \def\lt{<}
    % Document parameters
    \title{handbook}
    
    
    

    % Pygments definitions
    
\makeatletter
\def\PY@reset{\let\PY@it=\relax \let\PY@bf=\relax%
    \let\PY@ul=\relax \let\PY@tc=\relax%
    \let\PY@bc=\relax \let\PY@ff=\relax}
\def\PY@tok#1{\csname PY@tok@#1\endcsname}
\def\PY@toks#1+{\ifx\relax#1\empty\else%
    \PY@tok{#1}\expandafter\PY@toks\fi}
\def\PY@do#1{\PY@bc{\PY@tc{\PY@ul{%
    \PY@it{\PY@bf{\PY@ff{#1}}}}}}}
\def\PY#1#2{\PY@reset\PY@toks#1+\relax+\PY@do{#2}}

\expandafter\def\csname PY@tok@w\endcsname{\def\PY@tc##1{\textcolor[rgb]{0.73,0.73,0.73}{##1}}}
\expandafter\def\csname PY@tok@c\endcsname{\let\PY@it=\textit\def\PY@tc##1{\textcolor[rgb]{0.25,0.50,0.50}{##1}}}
\expandafter\def\csname PY@tok@cp\endcsname{\def\PY@tc##1{\textcolor[rgb]{0.74,0.48,0.00}{##1}}}
\expandafter\def\csname PY@tok@k\endcsname{\let\PY@bf=\textbf\def\PY@tc##1{\textcolor[rgb]{0.00,0.50,0.00}{##1}}}
\expandafter\def\csname PY@tok@kp\endcsname{\def\PY@tc##1{\textcolor[rgb]{0.00,0.50,0.00}{##1}}}
\expandafter\def\csname PY@tok@kt\endcsname{\def\PY@tc##1{\textcolor[rgb]{0.69,0.00,0.25}{##1}}}
\expandafter\def\csname PY@tok@o\endcsname{\def\PY@tc##1{\textcolor[rgb]{0.40,0.40,0.40}{##1}}}
\expandafter\def\csname PY@tok@ow\endcsname{\let\PY@bf=\textbf\def\PY@tc##1{\textcolor[rgb]{0.67,0.13,1.00}{##1}}}
\expandafter\def\csname PY@tok@nb\endcsname{\def\PY@tc##1{\textcolor[rgb]{0.00,0.50,0.00}{##1}}}
\expandafter\def\csname PY@tok@nf\endcsname{\def\PY@tc##1{\textcolor[rgb]{0.00,0.00,1.00}{##1}}}
\expandafter\def\csname PY@tok@nc\endcsname{\let\PY@bf=\textbf\def\PY@tc##1{\textcolor[rgb]{0.00,0.00,1.00}{##1}}}
\expandafter\def\csname PY@tok@nn\endcsname{\let\PY@bf=\textbf\def\PY@tc##1{\textcolor[rgb]{0.00,0.00,1.00}{##1}}}
\expandafter\def\csname PY@tok@ne\endcsname{\let\PY@bf=\textbf\def\PY@tc##1{\textcolor[rgb]{0.82,0.25,0.23}{##1}}}
\expandafter\def\csname PY@tok@nv\endcsname{\def\PY@tc##1{\textcolor[rgb]{0.10,0.09,0.49}{##1}}}
\expandafter\def\csname PY@tok@no\endcsname{\def\PY@tc##1{\textcolor[rgb]{0.53,0.00,0.00}{##1}}}
\expandafter\def\csname PY@tok@nl\endcsname{\def\PY@tc##1{\textcolor[rgb]{0.63,0.63,0.00}{##1}}}
\expandafter\def\csname PY@tok@ni\endcsname{\let\PY@bf=\textbf\def\PY@tc##1{\textcolor[rgb]{0.60,0.60,0.60}{##1}}}
\expandafter\def\csname PY@tok@na\endcsname{\def\PY@tc##1{\textcolor[rgb]{0.49,0.56,0.16}{##1}}}
\expandafter\def\csname PY@tok@nt\endcsname{\let\PY@bf=\textbf\def\PY@tc##1{\textcolor[rgb]{0.00,0.50,0.00}{##1}}}
\expandafter\def\csname PY@tok@nd\endcsname{\def\PY@tc##1{\textcolor[rgb]{0.67,0.13,1.00}{##1}}}
\expandafter\def\csname PY@tok@s\endcsname{\def\PY@tc##1{\textcolor[rgb]{0.73,0.13,0.13}{##1}}}
\expandafter\def\csname PY@tok@sd\endcsname{\let\PY@it=\textit\def\PY@tc##1{\textcolor[rgb]{0.73,0.13,0.13}{##1}}}
\expandafter\def\csname PY@tok@si\endcsname{\let\PY@bf=\textbf\def\PY@tc##1{\textcolor[rgb]{0.73,0.40,0.53}{##1}}}
\expandafter\def\csname PY@tok@se\endcsname{\let\PY@bf=\textbf\def\PY@tc##1{\textcolor[rgb]{0.73,0.40,0.13}{##1}}}
\expandafter\def\csname PY@tok@sr\endcsname{\def\PY@tc##1{\textcolor[rgb]{0.73,0.40,0.53}{##1}}}
\expandafter\def\csname PY@tok@ss\endcsname{\def\PY@tc##1{\textcolor[rgb]{0.10,0.09,0.49}{##1}}}
\expandafter\def\csname PY@tok@sx\endcsname{\def\PY@tc##1{\textcolor[rgb]{0.00,0.50,0.00}{##1}}}
\expandafter\def\csname PY@tok@m\endcsname{\def\PY@tc##1{\textcolor[rgb]{0.40,0.40,0.40}{##1}}}
\expandafter\def\csname PY@tok@gh\endcsname{\let\PY@bf=\textbf\def\PY@tc##1{\textcolor[rgb]{0.00,0.00,0.50}{##1}}}
\expandafter\def\csname PY@tok@gu\endcsname{\let\PY@bf=\textbf\def\PY@tc##1{\textcolor[rgb]{0.50,0.00,0.50}{##1}}}
\expandafter\def\csname PY@tok@gd\endcsname{\def\PY@tc##1{\textcolor[rgb]{0.63,0.00,0.00}{##1}}}
\expandafter\def\csname PY@tok@gi\endcsname{\def\PY@tc##1{\textcolor[rgb]{0.00,0.63,0.00}{##1}}}
\expandafter\def\csname PY@tok@gr\endcsname{\def\PY@tc##1{\textcolor[rgb]{1.00,0.00,0.00}{##1}}}
\expandafter\def\csname PY@tok@ge\endcsname{\let\PY@it=\textit}
\expandafter\def\csname PY@tok@gs\endcsname{\let\PY@bf=\textbf}
\expandafter\def\csname PY@tok@gp\endcsname{\let\PY@bf=\textbf\def\PY@tc##1{\textcolor[rgb]{0.00,0.00,0.50}{##1}}}
\expandafter\def\csname PY@tok@go\endcsname{\def\PY@tc##1{\textcolor[rgb]{0.53,0.53,0.53}{##1}}}
\expandafter\def\csname PY@tok@gt\endcsname{\def\PY@tc##1{\textcolor[rgb]{0.00,0.27,0.87}{##1}}}
\expandafter\def\csname PY@tok@err\endcsname{\def\PY@bc##1{\setlength{\fboxsep}{0pt}\fcolorbox[rgb]{1.00,0.00,0.00}{1,1,1}{\strut ##1}}}
\expandafter\def\csname PY@tok@kc\endcsname{\let\PY@bf=\textbf\def\PY@tc##1{\textcolor[rgb]{0.00,0.50,0.00}{##1}}}
\expandafter\def\csname PY@tok@kd\endcsname{\let\PY@bf=\textbf\def\PY@tc##1{\textcolor[rgb]{0.00,0.50,0.00}{##1}}}
\expandafter\def\csname PY@tok@kn\endcsname{\let\PY@bf=\textbf\def\PY@tc##1{\textcolor[rgb]{0.00,0.50,0.00}{##1}}}
\expandafter\def\csname PY@tok@kr\endcsname{\let\PY@bf=\textbf\def\PY@tc##1{\textcolor[rgb]{0.00,0.50,0.00}{##1}}}
\expandafter\def\csname PY@tok@bp\endcsname{\def\PY@tc##1{\textcolor[rgb]{0.00,0.50,0.00}{##1}}}
\expandafter\def\csname PY@tok@fm\endcsname{\def\PY@tc##1{\textcolor[rgb]{0.00,0.00,1.00}{##1}}}
\expandafter\def\csname PY@tok@vc\endcsname{\def\PY@tc##1{\textcolor[rgb]{0.10,0.09,0.49}{##1}}}
\expandafter\def\csname PY@tok@vg\endcsname{\def\PY@tc##1{\textcolor[rgb]{0.10,0.09,0.49}{##1}}}
\expandafter\def\csname PY@tok@vi\endcsname{\def\PY@tc##1{\textcolor[rgb]{0.10,0.09,0.49}{##1}}}
\expandafter\def\csname PY@tok@vm\endcsname{\def\PY@tc##1{\textcolor[rgb]{0.10,0.09,0.49}{##1}}}
\expandafter\def\csname PY@tok@sa\endcsname{\def\PY@tc##1{\textcolor[rgb]{0.73,0.13,0.13}{##1}}}
\expandafter\def\csname PY@tok@sb\endcsname{\def\PY@tc##1{\textcolor[rgb]{0.73,0.13,0.13}{##1}}}
\expandafter\def\csname PY@tok@sc\endcsname{\def\PY@tc##1{\textcolor[rgb]{0.73,0.13,0.13}{##1}}}
\expandafter\def\csname PY@tok@dl\endcsname{\def\PY@tc##1{\textcolor[rgb]{0.73,0.13,0.13}{##1}}}
\expandafter\def\csname PY@tok@s2\endcsname{\def\PY@tc##1{\textcolor[rgb]{0.73,0.13,0.13}{##1}}}
\expandafter\def\csname PY@tok@sh\endcsname{\def\PY@tc##1{\textcolor[rgb]{0.73,0.13,0.13}{##1}}}
\expandafter\def\csname PY@tok@s1\endcsname{\def\PY@tc##1{\textcolor[rgb]{0.73,0.13,0.13}{##1}}}
\expandafter\def\csname PY@tok@mb\endcsname{\def\PY@tc##1{\textcolor[rgb]{0.40,0.40,0.40}{##1}}}
\expandafter\def\csname PY@tok@mf\endcsname{\def\PY@tc##1{\textcolor[rgb]{0.40,0.40,0.40}{##1}}}
\expandafter\def\csname PY@tok@mh\endcsname{\def\PY@tc##1{\textcolor[rgb]{0.40,0.40,0.40}{##1}}}
\expandafter\def\csname PY@tok@mi\endcsname{\def\PY@tc##1{\textcolor[rgb]{0.40,0.40,0.40}{##1}}}
\expandafter\def\csname PY@tok@il\endcsname{\def\PY@tc##1{\textcolor[rgb]{0.40,0.40,0.40}{##1}}}
\expandafter\def\csname PY@tok@mo\endcsname{\def\PY@tc##1{\textcolor[rgb]{0.40,0.40,0.40}{##1}}}
\expandafter\def\csname PY@tok@ch\endcsname{\let\PY@it=\textit\def\PY@tc##1{\textcolor[rgb]{0.25,0.50,0.50}{##1}}}
\expandafter\def\csname PY@tok@cm\endcsname{\let\PY@it=\textit\def\PY@tc##1{\textcolor[rgb]{0.25,0.50,0.50}{##1}}}
\expandafter\def\csname PY@tok@cpf\endcsname{\let\PY@it=\textit\def\PY@tc##1{\textcolor[rgb]{0.25,0.50,0.50}{##1}}}
\expandafter\def\csname PY@tok@c1\endcsname{\let\PY@it=\textit\def\PY@tc##1{\textcolor[rgb]{0.25,0.50,0.50}{##1}}}
\expandafter\def\csname PY@tok@cs\endcsname{\let\PY@it=\textit\def\PY@tc##1{\textcolor[rgb]{0.25,0.50,0.50}{##1}}}

\def\PYZbs{\char`\\}
\def\PYZus{\char`\_}
\def\PYZob{\char`\{}
\def\PYZcb{\char`\}}
\def\PYZca{\char`\^}
\def\PYZam{\char`\&}
\def\PYZlt{\char`\<}
\def\PYZgt{\char`\>}
\def\PYZsh{\char`\#}
\def\PYZpc{\char`\%}
\def\PYZdl{\char`\$}
\def\PYZhy{\char`\-}
\def\PYZsq{\char`\'}
\def\PYZdq{\char`\"}
\def\PYZti{\char`\~}
% for compatibility with earlier versions
\def\PYZat{@}
\def\PYZlb{[}
\def\PYZrb{]}
\makeatother


    % Exact colors from NB
    \definecolor{incolor}{rgb}{0.0, 0.0, 0.5}
    \definecolor{outcolor}{rgb}{0.545, 0.0, 0.0}



    
    % Prevent overflowing lines due to hard-to-break entities
    \sloppy 
    % Setup hyperref package
    \hypersetup{
      breaklinks=true,  % so long urls are correctly broken across lines
      colorlinks=true,
      urlcolor=urlcolor,
      linkcolor=linkcolor,
      citecolor=citecolor,
      }
    % Slightly bigger margins than the latex defaults
    
    \geometry{verbose,tmargin=1in,bmargin=1in,lmargin=1in,rmargin=1in}
    
    

    \begin{document}
    
    
    \maketitle
    
    

    
    \hypertarget{hlab-analysis-methods-handbook}{%
\section{Hlab analysis methods
handbook}\label{hlab-analysis-methods-handbook}}

Welcome! This rather lengthy document summarizes the various topics we
will/have covered in the HLab analysis mini series. If this is the
.ipynb version of the doc, you can edit/run blocks of code yourself
(assuming you have Julia and the necessary packages installed).
Otherwise, you can open a new Julia session and follow along with the
code examples below

    \hypertarget{table-of-contents}{%
\subsection{Table of contents}\label{table-of-contents}}

\begin{enumerate}
\def\labelenumi{\arabic{enumi}.}
\setcounter{enumi}{-1}
\tightlist
\item
  Section \ref{basics}
\item
  Section \ref{linalg} 1.1 Section \ref{norm} 1.2 Section \ref{inner}
  1.3 Section \ref{outer} 1.4 Section \ref{rank}
\item
  Section \ref{sigproc}
\end{enumerate}

    \hypertarget{basic-setup}{%
\section{0 Basic Setup }\label{basic-setup}}

If you are new to Julia, fear not, it is quite similar to MATLAB. There
are, however a few peculiarities to take care of. First, in Julia we
need to \emph{import} packages to use certain functions. For instance,
there are no built-in plotting functions, but there is a handy
\emph{Plots} package that we can install and use. In general, to
increase the flexibility of Julia we first need to (a) install desired
packages, and (b) import them when we wish to use them.

For the most part, Julia code will look very familiar if you have
learned Python and/or MATLAB (it uses the best of both worlds: MATLAB's
intuitive linear algebra framework with Python's general-purposeness \&
modularity). Plotting is a bit different, as are some of the data
structures (for instance, a row-vector does not behave the same as a
column-vector, but we'll get into that next)

For now, we will need to install at least a few additional packages to
get up to speed. If you haven't already, in an active Julia session (or
in this notebook if you are using it in Jupyter) type the following
commands to get plotting, statistics, and tabular-based data structures
available for use:

    \begin{Verbatim}[commandchars=\\\{\}]
{\color{incolor}In [{\color{incolor} }]:} \PY{k}{using} \PY{n}{Pkg} \PY{c}{\PYZsh{} allows us to add new packages}
        \PY{n}{Pkg}\PY{o}{.}\PY{n}{update}\PY{p}{(}\PY{p}{)}
        \PY{n}{Pkg}\PY{o}{.}\PY{n}{add}\PY{p}{(}\PY{l+s}{\PYZdq{}}\PY{l+s}{P}\PY{l+s}{l}\PY{l+s}{o}\PY{l+s}{t}\PY{l+s}{s}\PY{l+s}{\PYZdq{}}\PY{p}{)}      \PY{c}{\PYZsh{} plotting capabilities}
        \PY{n}{Pkg}\PY{o}{.}\PY{n}{add}\PY{p}{(}\PY{l+s}{\PYZdq{}}\PY{l+s}{S}\PY{l+s}{t}\PY{l+s}{a}\PY{l+s}{t}\PY{l+s}{s}\PY{l+s}{P}\PY{l+s}{l}\PY{l+s}{o}\PY{l+s}{t}\PY{l+s}{s}\PY{l+s}{\PYZdq{}}\PY{p}{)} \PY{c}{\PYZsh{} statsitics\PYZhy{}related plotting add ons }
        \PY{n}{Pkg}\PY{o}{.}\PY{n}{add}\PY{p}{(}\PY{l+s}{\PYZdq{}}\PY{l+s}{D}\PY{l+s}{a}\PY{l+s}{t}\PY{l+s}{a}\PY{l+s}{F}\PY{l+s}{r}\PY{l+s}{a}\PY{l+s}{m}\PY{l+s}{e}\PY{l+s}{s}\PY{l+s}{\PYZdq{}}\PY{p}{)} \PY{c}{\PYZsh{} tabular data structures}
\end{Verbatim}


    We can use these packages once installed via the
\texttt{using\ \_\_package\_name\_\_} command, which imports all the
functions available from that particular package. Contrastingly,
importing via the \texttt{import\ \_\_package\_name\_\_} imports the
package, but you have to use dot (.) notation to access the functions.

Personally, I prefer the \texttt{using} command for packages that you
will almost certainly use throughout your code (for instance the
\texttt{Plots} package), and the \texttt{import} command for packages
that will be used more rarely. This can help you keep track of which
functions stem from which packages, and IMO keeps the code looking
cleaner.

    \begin{Verbatim}[commandchars=\\\{\}]
{\color{incolor}In [{\color{incolor}10}]:} \PY{k}{using} \PY{n}{Plots}
         \PY{n}{x} \PY{o}{=} \PY{n}{randn}\PY{p}{(}\PY{l+m+mi}{100}\PY{p}{)}
         \PY{n}{plot}\PY{p}{(}\PY{n}{x}\PY{p}{)}
         
         \PY{k}{import} \PY{n}{Statistics}
         \PY{n}{print}\PY{p}{(}\PY{p}{[}\PY{n}{Statistics}\PY{o}{.}\PY{n}{mean}\PY{p}{(}\PY{n}{x}\PY{p}{)}\PY{p}{,} \PY{n}{Statistics}\PY{o}{.}\PY{n}{std}\PY{p}{(}\PY{n}{x}\PY{p}{)}\PY{p}{]}\PY{p}{)}
\end{Verbatim}


    \begin{Verbatim}[commandchars=\\\{\}]
[-0.00331494, 1.05387]
    \end{Verbatim}

    \hypertarget{linear-algebra}{%
\section{1 Linear Algebra }\label{linear-algebra}}

Linear algbera is at the core of most (if not all) of the analysis
methods we will be covering. As such, knowing even a very surface-level
amount of linear algebra can be extremely rewarding/beneficial when
creating your analysis pipelines.

Let's begin by creating a simple \emph{vector} in Julia, which is
defined as a column-vector by default:

    \begin{Verbatim}[commandchars=\\\{\}]
{\color{incolor}In [{\color{incolor} }]:} \PY{n}{x} \PY{o}{=} \PY{p}{[}\PY{l+m+mi}{1}\PY{p}{,} \PY{l+m+mi}{2}\PY{p}{,} \PY{l+m+mi}{3}\PY{p}{,} \PY{l+m+mi}{4}\PY{p}{,} \PY{l+m+mi}{5}\PY{p}{]} \PY{c}{\PYZsh{} unlike matlab, you must use commas if you want to specify a column vector}
        \PY{n}{size}\PY{p}{(}\PY{n}{x}\PY{p}{)}
\end{Verbatim}


    We see that our variable \texttt{x} is a 5 x 1 column vector. This is an
important thing to note, since linear algebra cares about the
orientation of vectors/matrices. Of course, we can do simple things like
add/multiply a scalar. We can also add another vector of the same
length, or multiply by another vector or matrix.

    \begin{Verbatim}[commandchars=\\\{\}]
{\color{incolor}In [{\color{incolor} }]:} \PY{n}{y} \PY{o}{=} \PY{p}{[}\PY{l+m+mi}{0}\PY{p}{,} \PY{l+m+mi}{1}\PY{p}{,} \PY{l+m+mi}{0}\PY{p}{,} \PY{l+m+mi}{1}\PY{p}{,} \PY{l+m+mi}{2}\PY{p}{]}
        \PY{n}{print}\PY{p}{(}\PY{l+m+mi}{2}\PY{n}{x}\PY{p}{)} \PY{c}{\PYZsh{} multiply each element by 2}
        \PY{n}{print}\PY{p}{(}\PY{n}{x} \PY{o}{.*} \PY{n}{y}\PY{p}{)} \PY{c}{\PYZsh{} element\PYZhy{}wise multiplication}
\end{Verbatim}


    \hypertarget{vector-and-matrix-norms}{%
\subsubsection{1.1 Vector and matrix norms
}\label{vector-and-matrix-norms}}

We can make some measurements on vectors and matrices that can come in
handy for things such as scaling data (useful/necessary for some matrix
factorization and clustering techniques) and analyzing goodness-of-fit
for regression problems.

The \emph{norm} of a vector is defined as:
\[\left|\left|v\right|\right|^{p} = \sum{v_i^p}\]

So the \emph{2-norm} of a vector is the \emph{sum-of-squares} (because p
-\textgreater{} 2). The norm of a matrix is the same thing except that
we sum elements over both rows and columns.

Norms come in handy for scaling data (divide each vector by its norm),
which can improve clustering/dimensionality reduction techniques since
all variables (column of a matrix) become similar in magnitude, and thus
no single variable dominates the clustering/dimensionality reduction
solution. Moreover, the \emph{2-norm} (also called the \emph{L2 norm})
is used in regression. The term \emph{least squares} comes directly from
trying to minimize the \emph{L2 norm} of the residuals following a
regression fit (i.e.~minimize the \emph{sum of squares} of the errors).

    \hypertarget{the-inner-product}{%
\subsubsection{1.2 The inner product }\label{the-inner-product}}

We can think of the \emph{L2 norm} as an \emph{inner product} (or
\emph{dot product}) of a vector with itself. An inner product between
vectors \texttt{x} and \texttt{y} is defined as the sum of element-wise
products.

For instance, given vectors \texttt{x} and \texttt{y}, the dot product
is:

\begin{Shaded}
\begin{Highlighting}[]
\NormalTok{x' * y}
\end{Highlighting}
\end{Shaded}

From above, you can see that I had to transpose the vector \texttt{x}
first for this to work. I also could have imported the
\texttt{LinearAlgebra} package and used the function
\texttt{LinearAlgebra.dot(x,\ y)}.

We can write our own dot-product function to clarify what it is doing:

    \begin{Verbatim}[commandchars=\\\{\}]
{\color{incolor}In [{\color{incolor} }]:} \PY{n}{dot}\PY{p}{(}\PY{n}{x}\PY{p}{,} \PY{n}{y}\PY{p}{)} \PY{o}{=} \PY{n}{sum}\PY{p}{(}\PY{p}{[}\PY{n}{i}\PY{o}{*}\PY{n}{j} \PY{k}{for} \PY{n}{i} \PY{k+kp}{in} \PY{n}{x}\PY{p}{,} \PY{k}{for} \PY{n}{j} \PY{k+kp}{in} \PY{n}{y}\PY{p}{]}\PY{p}{)} 
        \PY{n}{print}\PY{p}{(}\PY{n}{x}\PY{o}{\PYZsq{}} \PY{o}{*} \PY{n}{y}\PY{p}{)}    \PY{c}{\PYZsh{} built\PYZhy{}in dot product}
        \PY{n}{print}\PY{p}{(}\PY{n}{dot}\PY{p}{(}\PY{n}{x}\PY{p}{,} \PY{n}{y}\PY{p}{)}\PY{p}{)} \PY{c}{\PYZsh{} homemade dot product}
\end{Verbatim}


    So the inner/dot product is nothing more than summing the resulting
vector after doing element wise multiplication. If we have a
matrix-vector product or matrix-matrix product, we simply repeat the
above procedure for each row/column of the matrices involved.

Inner products depend on a certain order, so if we have two matrices
\texttt{A} and \texttt{B}, the inner product \texttt{A\ *\ B} is not the
same as \texttt{B\ *\ A}. The former performs a dot product between each
\emph{row of A} and each \emph{column of B}, while the latter does the
opposite.

Inner products are at the core of much of linear algebra, and indeed
many of the analyses we will cover. It can be interpreted in a variety
of ways, some of which are more useful than others depending on the
context:

\begin{enumerate}
\def\labelenumi{\arabic{enumi}.}
\tightlist
\item
  A similarity score (clustering)
\item
  A projection of one vector/matrix onto another (dim reduction methods,
  discriminant analysis)
\item
  A weighted mixture (demixture) of factors (observations) (factor
  analysis, PCA)
\item
  A weighted sum of predictor variables (regression)
\item
  A rotation/scaling of data (variety of signal/image processing
  methods)
\item
  A filter response (signal processing/convolution)
\item
  A transformation of \emph{inputs} into \emph{outputs} (variety of
  topics/methods, such as neural networks)
\end{enumerate}

I usually find the inner product easiest to think of in terms of (1),
(3), and/or (4), as these are likely to be encountered in your analysis,
but it can be useful to visualize/interpret inner products as
transformations of vectors/variables:

    \begin{Verbatim}[commandchars=\\\{\}]
{\color{incolor}In [{\color{incolor} }]:} \PY{c}{\PYZsh{} create a correlated scatter plot}
        \PY{n}{x} \PY{o}{=} \PY{n}{randn}\PY{p}{(}\PY{l+m+mi}{1000}\PY{p}{)}
        \PY{n}{y} \PY{o}{=} \PY{l+m+mi}{3}\PY{n}{x} \PY{o}{+} \PY{l+m+mf}{0.5}\PY{o}{*}\PY{n}{randn}\PY{p}{(}\PY{l+m+mi}{1000}\PY{p}{)} 
        \PY{n}{data} \PY{o}{=} \PY{p}{[}\PY{n}{x}\PY{p}{,} \PY{n}{y}\PY{p}{]}
        \PY{n}{plot}\PY{p}{(}\PY{n}{x}\PY{p}{,} \PY{n}{y}\PY{p}{,} \PY{n}{seriestype}\PY{o}{=}\PY{l+s}{\PYZdq{}}\PY{l+s}{s}\PY{l+s}{c}\PY{l+s}{a}\PY{l+s}{t}\PY{l+s}{t}\PY{l+s}{e}\PY{l+s}{r}\PY{l+s}{\PYZdq{}}\PY{p}{)}
        
        \PY{c}{\PYZsh{} now lets flip over the y\PYZhy{}axis using an inner product with a rotation matrix}
        \PY{n}{A} \PY{o}{=} \PY{p}{[}\PY{l+m+mi}{1} \PY{l+m+mi}{0}\PY{p}{;} \PY{l+m+mi}{0} \PY{o}{\PYZhy{}}\PY{l+m+mi}{1}\PY{p}{]}
        \PY{n}{data\PYZus{}filped} \PY{o}{=} \PY{n}{data} \PY{o}{*} \PY{n}{A}
        \PY{n}{plot!}\PY{p}{(}\PY{n}{data\PYZus{}flipped}\PY{p}{[}\PY{o}{:}\PY{p}{,}\PY{l+m+mi}{1}\PY{p}{]}\PY{p}{,} \PY{n}{data\PYZus{}flipped}\PY{p}{[}\PY{o}{:}\PY{p}{,}\PY{l+m+mi}{2}\PY{p}{]}\PY{p}{,} \PY{n}{seriestype}\PY{o}{=}\PY{l+s}{\PYZdq{}}\PY{l+s}{s}\PY{l+s}{c}\PY{l+s}{a}\PY{l+s}{t}\PY{l+s}{t}\PY{l+s}{e}\PY{l+s}{r}\PY{l+s}{\PYZdq{}}\PY{p}{)}
\end{Verbatim}


    As you can see above, we have used the inner product in a
\emph{rotation/transformation} setting, where we've flipped our data
about the y-axis. It is helpful to understand the structure of
\texttt{A}, and what it is doing to each of our original
\emph{dimensions} (each column of \texttt{data}). Our rotation matrix
\texttt{A} contains the values:

\begin{verbatim}
1  0
0 -1
\end{verbatim}

Each \emph{row} of \texttt{A} is associated with one
dimension/feature/variable of the \emph{input} data, while each
\emph{column} of \texttt{A} is associated with one
dimension/feature/variable of the \emph{ouput} data. Since we have two
columns in \texttt{data}, and two columns in \texttt{A}, we already know
that this inner product is NOT going to change the dimensionality of the
data. We started with two variables (x, y), and we will be ending with
two variables (x\_new, y\_new).

Looking at the \emph{columns} of A, we see that the first column,
\texttt{{[}1,\ 0{]}}, will be multiplying the \emph{first} variable in
\texttt{data} (x) by the value \emph{1}, and the \emph{second} variable
in \texttt{data} (y) by the value \emph{0}, then adding the results
together to produce \texttt{x\_new}. Since we are not combining
\texttt{x} and \texttt{y} together (since y is multiplied by 0), and
since we are only multiplying \texttt{x} by 1, we know that
\texttt{x\_new\ =\ x}.

Similarly, the second column of \texttt{A}, \texttt{{[}0,\ -1{]}}, will
be multiplying \texttt{x} by \emph{0}, and \texttt{y} by \emph{-1}, then
summing the results to produce \texttt{y\_new}. Since the only
contribution to \texttt{y\_new} is the original \texttt{y} variable in
\texttt{data}, and it is only multiplied by \emph{-1}, we know that
\texttt{y\_new\ =\ -y}.

So in general, each \emph{column} of \texttt{A} specifies how the input
variables get weighted before being summed together to produce a new,
single, ouput variable. So each column of \texttt{A} is performing a
\emph{linear combination} of the input variables. We can also readily
appreciate how we can easily shrink or expand the original
dimensionality of the data set (by removing or adding columns to
\texttt{A}, respectively).

    \hypertarget{the-outer-product}{%
\subsubsection{1.3 The outer product }\label{the-outer-product}}

Unlike the inner product, which typically reduces the \# of elements
(for instance, from two vectors -\textgreater{} scalar), the outer
product is an \emph{expansion} of a dataset. You will likely not use an
outer product as often as you will an inner product, but it has
important applications, especially relating to dimensionality reduction
techniques and data reconstruction.

If we have two column vectors \texttt{x} and \texttt{y}, we perform an
outer product as \texttt{x\ *\ y\textquotesingle{}}. Note that while we
transposed \texttt{x} for an inner product, here we have transposed
\texttt{y}. The result is that we take the vector \texttt{x}, and
multiply all of its elements by each element in \texttt{y}. Basically,
we are simply copying the vector \texttt{x} many times, and scaling each
copy depending on the value of each element in \texttt{y}. The result is
a matrix that just contains scaled copies of \texttt{x}.

It is easer to understand this operation with a quick example:

    \begin{Verbatim}[commandchars=\\\{\}]
{\color{incolor}In [{\color{incolor} }]:} \PY{n}{x} \PY{o}{=} \PY{n}{sin}\PY{o}{.}\PY{p}{(}\PY{n}{range}\PY{p}{(}\PY{l+m+mi}{0}\PY{p}{,} \PY{n}{stop}\PY{o}{=}\PY{l+m+mi}{10}\PY{p}{,} \PY{n}{length}\PY{o}{=}\PY{l+m+mi}{100}\PY{p}{)}\PY{p}{)} \PY{c}{\PYZsh{} sine wave 100 pts long}
        \PY{n}{w} \PY{o}{=} \PY{p}{[}\PY{l+m+mi}{1}\PY{p}{,} \PY{l+m+mi}{2}\PY{p}{,} \PY{l+m+mf}{0.5}\PY{p}{,} \PY{l+m+mi}{4}\PY{p}{,} \PY{o}{\PYZhy{}}\PY{l+m+mi}{3}\PY{p}{]} \PY{c}{\PYZsh{} weights for each \PYZdq{}copy\PYZdq{} of x}
        
        \PY{n}{M} \PY{o}{=} \PY{n}{x} \PY{o}{*} \PY{n}{w}\PY{o}{\PYZsq{}} \PY{c}{\PYZsh{} now this is of size 100 x 5, since x was 100 x 1 and w\PYZsq{} is 1 x 5}
        \PY{n}{heatmap}\PY{p}{(}\PY{n}{M}\PY{p}{)}
\end{Verbatim}


    As you can see, each column in \texttt{M} is simply \texttt{x} that has
been scaled according to the corresponding element in \texttt{w}. Even
though we have 5 columns in \texttt{M}, really each column can be
represented as the original sine wave with a scaling, so that the matrix
\texttt{M} is actually a very simple matrix (with \texttt{rank\ =\ 1}
\ldots{} more on that below).

This application comes up when we use dimensionality reduction
techniques like PCA, and wish to get a \emph{reconstruction} of our
original data using some (or all) of the principal components. The idea
is that we can decompose our data matrices into a bunch of ``simple''
factors and reconstruct the data by performing outer products using the
factors and summing the result.

    \hypertarget{matrix-rank}{%
\subsubsection{1.4 Matrix rank }\label{matrix-rank}}

As an explicit example of the above, lets demonstrate how we can build a
rather complex looking dataset using just a few simple functions and
some outer products. This is a toy example that demonstrates the way one
may reconstruct data from principal components etc.

    \begin{Verbatim}[commandchars=\\\{\}]
{\color{incolor}In [{\color{incolor} }]:} \PY{c}{\PYZsh{} build our functions}
        \PY{n}{f1}\PY{p}{(}\PY{n}{x}\PY{p}{)} \PY{o}{=} \PY{n}{sin}\PY{o}{.}\PY{p}{(}\PY{n}{x}\PY{p}{)}
        \PY{n}{f2}\PY{p}{(}\PY{n}{x}\PY{p}{)} \PY{o}{=} \PY{n}{cos}\PY{o}{.}\PY{p}{(}\PY{n}{x}\PY{p}{)}
        \PY{n}{f3}\PY{p}{(}\PY{n}{x}\PY{p}{)} \PY{o}{=} \PY{o}{\PYZhy{}}\PY{l+m+mf}{0.5}\PY{n}{x} \PY{o}{+} \PY{n}{x}\PY{o}{.\PYZca{}}\PY{l+m+mf}{0.5}
        
        \PY{c}{\PYZsh{} create our \PYZdq{}true\PYZdq{} simple functions. Each one may, for instance, represent a discovered}
        \PY{c}{\PYZsh{} \PYZdq{}factor\PYZdq{} from a dataset that expalains a good amount of variability in the data (a la PCA)}
        \PY{n}{x} \PY{o}{=} \PY{n}{range}\PY{p}{(}\PY{l+m+mi}{0}\PY{p}{,} \PY{n}{stop}\PY{o}{=}\PY{l+m+mi}{10}\PY{p}{,} \PY{n}{length}\PY{o}{=}\PY{l+m+mi}{100}\PY{p}{)}
        \PY{n}{factors} \PY{o}{=} \PY{p}{[}\PY{n}{f1}\PY{p}{(}\PY{n}{x}\PY{p}{)}\PY{p}{,} \PY{n}{f2}\PY{p}{(}\PY{n}{x}\PY{p}{)}\PY{p}{,} \PY{n}{f3}\PY{p}{(}\PY{n}{x}\PY{p}{)}\PY{p}{]} \PY{c}{\PYZsh{} 100 x 3 matrix}
        
        \PY{c}{\PYZsh{} create our \PYZdq{}weights\PYZdq{}, which determine how we the simple functions}
        \PY{c}{\PYZsh{} are scaled for each of our \PYZdq{}measurements\PYZdq{} that we made}
        \PY{n}{weights} \PY{o}{=} \PY{n}{rand}\PY{p}{(}\PY{l+m+mi}{10}\PY{p}{,} \PY{l+m+mi}{3}\PY{p}{)} \PY{c}{\PYZsh{} each factor will be scaled randomly 10 times; the 10 here could represent 10 different measurements or cells or something}
        
        \PY{c}{\PYZsh{} produce the full, observed dataset by scaling the factors with the weights}
        \PY{n}{data} \PY{o}{=} \PY{n}{factors} \PY{o}{*} \PY{n}{weights}\PY{o}{\PYZsq{}}
        \PY{n}{p1} \PY{o}{=} \PY{n}{plot}\PY{p}{(}\PY{n}{factors}\PY{p}{,} \PY{n}{title}\PY{o}{=}\PY{l+s}{\PYZdq{}}\PY{l+s}{t}\PY{l+s}{r}\PY{l+s}{u}\PY{l+s}{e}\PY{l+s}{ }\PY{l+s}{f}\PY{l+s}{a}\PY{l+s}{c}\PY{l+s}{t}\PY{l+s}{o}\PY{l+s}{r}\PY{l+s}{s}\PY{l+s}{\PYZdq{}}\PY{p}{)}
        \PY{n}{p2} \PY{o}{=} \PY{n}{heatmap}\PY{p}{(}\PY{n}{weights}\PY{p}{,} \PY{n}{title}\PY{o}{=}\PY{l+s}{\PYZdq{}}\PY{l+s}{w}\PY{l+s}{e}\PY{l+s}{i}\PY{l+s}{g}\PY{l+s}{h}\PY{l+s}{t}\PY{l+s}{ }\PY{l+s}{m}\PY{l+s}{a}\PY{l+s}{t}\PY{l+s}{r}\PY{l+s}{i}\PY{l+s}{x}\PY{l+s}{\PYZdq{}}\PY{p}{)}
        \PY{n}{p3} \PY{o}{=} \PY{n}{heatmap}\PY{p}{(}\PY{n}{data}\PY{p}{,} \PY{n}{title}\PY{o}{=}\PY{l+s}{\PYZdq{}}\PY{l+s}{o}\PY{l+s}{b}\PY{l+s}{s}\PY{l+s}{e}\PY{l+s}{r}\PY{l+s}{v}\PY{l+s}{e}\PY{l+s}{d}\PY{l+s}{ }\PY{l+s}{d}\PY{l+s}{a}\PY{l+s}{t}\PY{l+s}{a}\PY{l+s}{\PYZdq{}}\PY{p}{)}
        \PY{n}{plot}\PY{p}{(}\PY{n}{p1}\PY{p}{,} \PY{n}{p2}\PY{p}{,} \PY{n}{p3}\PY{p}{,} \PY{n}{layout} \PY{o}{=} \PY{n+nd}{@layout} \PY{p}{[}\PY{n}{a} \PY{n}{b}\PY{p}{;} \PY{n}{c}\PY{p}{]}\PY{p}{)}
\end{Verbatim}


    What we have done above is taken an \emph{inner product} of our true
factors with a weighting matrix. However, we can break this done as a
\emph{sum of outer products}, where each factor gets called randomly
using an out product, then the resulting matrices are all summed
together:

    \begin{Verbatim}[commandchars=\\\{\}]
{\color{incolor}In [{\color{incolor} }]:} \PY{n}{data\PYZus{}1} \PY{o}{=} \PY{n}{factors}\PY{p}{[}\PY{o}{:}\PY{p}{,}\PY{l+m+mi}{1}\PY{p}{]} \PY{o}{*} \PY{n}{weights}\PY{p}{[}\PY{o}{:}\PY{p}{,}\PY{l+m+mi}{1}\PY{p}{]}\PY{o}{\PYZsq{}} \PY{c}{\PYZsh{} outer product of factor 1 with weights for factor 1}
        \PY{n}{data\PYZus{}2} \PY{o}{=} \PY{n}{factors}\PY{p}{[}\PY{o}{:}\PY{p}{,}\PY{l+m+mi}{2}\PY{p}{]} \PY{o}{*} \PY{n}{weights}\PY{p}{[}\PY{o}{:}\PY{p}{,}\PY{l+m+mi}{2}\PY{p}{]}\PY{o}{\PYZsq{}} \PY{c}{\PYZsh{} ditto, for factor 2}
        \PY{n}{data\PYZus{}3} \PY{o}{=} \PY{n}{factors}\PY{p}{[}\PY{o}{:}\PY{p}{,}\PY{l+m+mi}{3}\PY{p}{]} \PY{o}{*} \PY{n}{weights}\PY{p}{[}\PY{o}{:}\PY{p}{,}\PY{l+m+mi}{3}\PY{p}{]}\PY{o}{\PYZsq{}} \PY{c}{\PYZsh{} ditto, for factor 3}
        \PY{n}{data\PYZus{}outer} \PY{o}{=} \PY{n}{data\PYZus{}1} \PY{o}{+} \PY{n}{data\PYZus{}2} \PY{o}{+} \PY{n}{data\PYZus{}3}
        
        \PY{n}{p1} \PY{o}{=} \PY{n}{heatmap}\PY{p}{(}\PY{n}{data\PYZus{}1}\PY{p}{,} \PY{n}{title}\PY{o}{=}\PY{l+s}{\PYZdq{}}\PY{l+s}{f}\PY{l+s}{a}\PY{l+s}{c}\PY{l+s}{t}\PY{l+s}{o}\PY{l+s}{r}\PY{l+s}{ }\PY{l+s}{1}\PY{l+s}{ }\PY{l+s}{m}\PY{l+s}{a}\PY{l+s}{t}\PY{l+s}{r}\PY{l+s}{i}\PY{l+s}{x}\PY{l+s}{\PYZdq{}}\PY{p}{)}
        \PY{n}{p2} \PY{o}{=} \PY{n}{heatmap}\PY{p}{(}\PY{n}{data\PYZus{}2}\PY{p}{,} \PY{n}{title}\PY{o}{=}\PY{l+s}{\PYZdq{}}\PY{l+s}{f}\PY{l+s}{a}\PY{l+s}{c}\PY{l+s}{t}\PY{l+s}{o}\PY{l+s}{r}\PY{l+s}{ }\PY{l+s}{2}\PY{l+s}{ }\PY{l+s}{m}\PY{l+s}{a}\PY{l+s}{t}\PY{l+s}{r}\PY{l+s}{i}\PY{l+s}{x}\PY{l+s}{\PYZdq{}}\PY{p}{)}
        \PY{n}{p3} \PY{o}{=} \PY{n}{heatmap}\PY{p}{(}\PY{n}{data\PYZus{}3}\PY{p}{,} \PY{n}{title}\PY{o}{=}\PY{l+s}{\PYZdq{}}\PY{l+s}{f}\PY{l+s}{a}\PY{l+s}{c}\PY{l+s}{t}\PY{l+s}{o}\PY{l+s}{r}\PY{l+s}{ }\PY{l+s}{3}\PY{l+s}{ }\PY{l+s}{m}\PY{l+s}{a}\PY{l+s}{t}\PY{l+s}{r}\PY{l+s}{i}\PY{l+s}{x}\PY{l+s}{\PYZdq{}}\PY{p}{)}
        \PY{n}{p4} \PY{o}{=} \PY{n}{heatmap}\PY{p}{(}\PY{n}{data\PYZus{}outer}\PY{p}{,} \PY{n}{title}\PY{o}{=}\PY{l+s}{\PYZdq{}}\PY{l+s}{o}\PY{l+s}{b}\PY{l+s}{s}\PY{l+s}{e}\PY{l+s}{r}\PY{l+s}{v}\PY{l+s}{e}\PY{l+s}{d}\PY{l+s}{ }\PY{l+s}{d}\PY{l+s}{a}\PY{l+s}{t}\PY{l+s}{a}\PY{l+s}{\PYZdq{}}\PY{p}{)}
        \PY{n}{plot}\PY{p}{(}\PY{n}{p1}\PY{p}{,} \PY{n}{p2}\PY{p}{,} \PY{n}{p3}\PY{p}{,} \PY{n}{p4}\PY{p}{,} \PY{n}{layout} \PY{o}{=} \PY{n+nd}{@layout} \PY{p}{[}\PY{n}{a} \PY{n}{b} \PY{n}{c}\PY{p}{;} \PY{n}{d}\PY{p}{]}\PY{p}{)}
\end{Verbatim}


    So as you can see from above, performing an outer product for each
factor with its weight vector and summing the result is equivalent to
taking a weighted linear combination for each element across the
factors. The former is just a longer way of doing the latter.
Ultimately, both produce a rather complicated looking matrix, while in
reality it is simply composed of three simple functions that have been
scaled slightly differently across ``measurements'' and summed together.
The benefit of the more explicit outer-product-summation based method is
that we can observe what each factor matrix looks like and see that in
reality it is just a repeat of a simple function.

The three factor matrices above all have a \emph{rank} equal to
\textbf{1}. The \emph{rank} of a matrix tells us how many
\emph{independent columns} it contains. Independence here means that, if
we imagine each column representing some vector pointing in some
direction, independent vectors (columns) are those that make up a unique
(hyper)plane. So all of our factor matrices above have rank 1 because
the columns are all simply scaled versions of one another. So, in vector
space, all the vectors (columns) are lying on top of one another
(pointing in the same direction).

In contrast, our final observed matrix \texttt{data} has \emph{rank}
equal to \textbf{3}, because it is composed of three independent
factors. However, if we had instead made one of the factors simply a
copy (maybe scaled) of one of the other factors, then we would have
reduced the rank of \texttt{data} to \textbf{2}. The point of many
dimensionality reduction techniques is to find a \emph{low-rank}
approximation to a dataset under the assumption that the original data
may have high rank due to \emph{noise}.

    \hypertarget{signal-processing}{%
\section{Signal Processing }\label{signal-processing}}


    % Add a bibliography block to the postdoc
    
    
    
    \end{document}
